\documentclass[a4paper,12pt]{elsarticle_rfabbri}

%%%%%%%%%%%%%%%%%%%%%%%%%%%%%%%%%%%%%%%%%%%%%%%%%%
% prints all equation environments and nothing else, each cropped, one to a page.
%\usepackage[active,tightpage]{preview}
%\PreviewEnvironment{equation*}
%%%%%%%%%%%%%%%%%%%%%%%%%%%%%%%%%%%%%%%%%%%%%%%%%%
%\usepackage{xiiiemc}
\usepackage[small,bf]{caption}
\usepackage{graphicx}
\usepackage[usenames,dvipsnames]{color}
\usepackage{xcolor}
\usepackage{amsfonts}
\usepackage{amssymb}
\usepackage{amsmath}
\usepackage{indentfirst}
\usepackage{natbib}
\usepackage{fancyhdr}
%\usepackage[brazil]{babel}  				 					% CONVERSÃO EM PORTUGUÊS
\usepackage[utf8]{inputenc}                              	  % CONVERSÃO EM PORTUGUÊS
\usepackage{bbm} 								%para escrever double 1
\usepackage{algorithm}						% para escrever algoritmos
\usepackage{algpseudocode}				%para escrever pseudo-códigos
\usepackage{subfigure} 							%poe varias figuras como se fossem uma só
\usepackage{url}	
\usepackage[bookmarks,pdfborder={0 0 0}]{hyperref} % cria índice com referencias no arquivo pdf, 
\usepackage[top=2.0cm, bottom=3.3cm, left=2.5cm,right=2.5cm]{geometry} % Margens
\usepackage{setspace}    					     % para colocar espaçamento entre linhas
\definecolor{seagreen}{RGB}{46,139,87}
\usepackage{scalefnt}
\usepackage{listings}

\lstset{
  extendedchars=\true,
  inputencoding=utf8,
  language=Matlab,
  %showstringspaces=false,
  formfeed=\newpage,
  tabsize=4,
  %commentstyle=\itshape,
  basicstyle=\ttfamily\scriptsize,
  %basicstyle={\small\fontfamily{fvm}\fontseries{m}\selectfont},
  commentstyle=\color{Apricot}\bfseries,
  %commentstyle=\color{red}\itshape,
  stringstyle=\color{red},
  identifierstyle=\color{PineGreen},
  showstringspaces=false,
  keywordstyle=\color{blue}\bfseries,
  moredelim=[il][\large\textbf]{\#\# },
  morekeywords={models,range},
  numbers=left,
  numbersep=2pt,
  numberstyle=\tiny,%\color{blue}\bfseries,
  literate=%
  {ã}{{\~a}}1
  {â}{{\^a}}1
  {õ}{{\~o}}1
  {á}{{\'a}}1
  {ú}{{\'u}}1
  {í}{{\'i}}1
  {é}{{\'e}}1
  {Ç}{{\c{C}}}1
  {Õ}{{\~O}}1
  {Ê}{{\^E}}1
  {ó}{{\'o}}1
  {à}{{\`a}}1
  {Â}{{\^A}}1
  {ô}{{\^o}}1
  {ê}{{\^e}}1
  {ç}{{\c{c}}}1
}
%%%%%%%%%%%%%%%%%%%%%%%%%%%%%%%%%%%%%%%%
% You have two versions of the macro
% \draftnote{My note}. The first version puts notes (e.g. My note in the example)
% into the margin of your document. The second formats the note as nothing. You
% 'comment out' the version of the macro you don't want (using a % at the
% beginning of the line).
\newcommand{\draftnote}[1]{\marginpar{\tiny\raggedright\textsf{\hspace{0pt}#1}}}
%\newcommand{\draftnote}[1]{}

% This one is just for the comments for in-line text.
\newcommand{\indraftnote}[1]{\textcolor[HTML]{114406}{\texttt{\footnotesize[#1]}}}
%\newcommand{\indraftnote}[1]{}
\newcommand{\todo}[1]{\indraftnote{todo: #1}}

%###########################################
%  outros comandos

\providecommand{\sin}{} \renewcommand{\sin}{\hspace{2pt}\mathrm{sen}}
\providecommand{\tan}{} \renewcommand{\tan}{\hspace{2pt}\mathrm{tg}}
\newcommand{\ie}{{\it i.e.}}
\newcommand{\etc}{{\it etc}}
\newcommand{\eg}{{\it e.g.}}
\newcommand{\keywordsname}{Keywords}
\newenvironment{keywords}{%
\noindent
        {\em \bfseries \keywordsname: }%
         \rm }
      {\endquotation \vskip 12bp}

\newcommand{\code}[2]{
 \vspace{1em}
 \subsubsection*{#1}
 \lstinputlisting{#2}
}


\begin{document}

\begin{frontmatter}

\title{Real-Time Perceptually-Tuned Color Detection} 



\author[iprj,aa]{Ricardo Fabbri\corref{corr1}}
\ead{rfabbri@iprj.uerj.br}
\ead[url]{www.lems.brown.edu/~rfabbri}


\author[aa]{Gilson Beck}

\author[ifsc,aa]{Vilson Vieira}
\ead{vilson@void.cc}

\author[icmc,aa]{Fernando Gorodscy}

\author[iprj]{Marcos Oliveira Couto Filho}

\author[ifsc,aa]{Renato Fabbri}

\address[iprj]{Instituto Polit\'{e}cnico, Universidade do Estado do Rio de
Janeiro\\C.P.: 97282 - 28601-970 - Nova Friburgo, RJ, Brazil}

\address[ifsc]{Instituto de F\'{i}sica de S\~{a}o Carlos (IFSC), Universidade de
S\~{a}o Paulo (USP)\\ Av.  Trabalhador S\~{a}o Carlense, 400,
13560-970 - S\~{a}o Carlos, SP, Brazil}

\address[icmc]{Instituto de Ci\^{e}ncias Matem\'{a}ticas e de
Computa\c{c}\~{a}o (ICMC), Universidade de
S\~{a}o Paulo (USP)\\ Av.  Trabalhador S\~{a}o Carlense, 400,
13560-970 - S\~{a}o Carlos, SP, Brazil}

\address[aa] {LabMacambira.sourceforge.net distributed hacker group}

\cortext[corr1]{Corresponding Author.\  Fax: +55 22 2533 5149}


%\thispagestyle{fancy}

\begin{abstract}
Color detector 
\end{abstract}

\begin{keyword}
Color detection\sep machine learning \sep interactive
computer vision \sep real time \sep photometric tracking \sep segmentation
\end{keyword}

\end{frontmatter}


\section{Introduction}

\begin{verbatim}
- color detector of 6 predefined classes in the context of trakcing moving
objects made of paper in an art installation
- skin color obtained as side effect
- fine tuned to match human perception, as shown  in experiments
- basically uses a decision tree on HSV color space.
-- TODO: decision trees are the correct term?
\end{verbatim}

\subsection{Uses of color detection}
- solve segmentation problem
- structure from motion: enables to track objects



\subsection{Previous work}
\textit{\textbf{Why is there a need for a new approach to color detection, given that so many
hard problems have been solved in computer vision in the last decade?}}



\section{Engineering the Detector} 

\subsection{Design}
Requirements: real-time, simplicity.


\subsection{Training and Fine-Tuning through Evaluation}

The system was trained and fine-tuned in Scilab~\cite{Fabbri:etal:Arxiv2012}. It
was then converted to C for use within the Pure Data real time dataflow
framework.

PS3Eye-specific.

\paragraph{A new ``origami'' color paper dataset}
The dataset is available on-line at..

\subsection{Execution Procedure}

\subsection{Color Balance}
We start off with a color calibration board in order to lock in the adaptive white
balance of the camera. This is merely a white sheet with small samples of the
actual color paper to be used in the application. We don't alter the hardware of
the camera as we want to keep the system easy to use by the non-specialist (\eg,
a musician or arts performer). Work is in progress to hack open source camera
drivers in order to provide hardware white balance locking and disabling of
the PS3Eye camera. However, the present procedure of white balance
calibration is simpler and will work with any cheap video camera.

\section{Experimental Results}

\subsection{Main Results and Evaluation}
\subsection{Comparison with Other Approaches}


\section{Conclusions}

\begin{verbatim}
- system extensively used in actual concerts and art installations, for
controlling sound
- it iproves state of the art in being very simple and very accurate
\end{verbatim}

\section*{Acknowledgements}
We thank Teia, casa de criacao, who initially funded the development of the proposed
system.  The authors also thank the Brazilian agencies CNPq, CAPES, FAPESP, and
FAPERJ for later financial support.

\bibliographystyle{elsarticle-num}
\bibliography{personal}

\end{document}
\endinput
