%%%%%%%%%%%%%%%%%%%%%%%%%%%%%%%%%%%%%%%%%%%%%%%%%%
% prints all equation environments and nothing else, each cropped, one to a page.
%\usepackage[active,tightpage]{preview}
%\PreviewEnvironment{equation*}
%%%%%%%%%%%%%%%%%%%%%%%%%%%%%%%%%%%%%%%%%%%%%%%%%%
%\usepackage{xiiiemc}
\usepackage[small,bf]{caption}
\usepackage{graphicx}
\usepackage[usenames,dvipsnames]{color}
\usepackage{xcolor}
\usepackage{amsfonts}
\usepackage{amssymb}
\usepackage{amsmath}
\usepackage{indentfirst}
\usepackage{natbib}
\usepackage{fancyhdr}
%\usepackage[brazil]{babel}  				 					% CONVERSÃO EM PORTUGUÊS
\usepackage[utf8]{inputenc}                              	  % CONVERSÃO EM PORTUGUÊS
\usepackage{bbm} 								%para escrever double 1
\usepackage{algorithm}						% para escrever algoritmos
\usepackage{algpseudocode}				%para escrever pseudo-códigos
\usepackage{subfigure} 							%poe varias figuras como se fossem uma só
\usepackage{url}	
\usepackage[bookmarks,pdfborder={0 0 0}]{hyperref} % cria índice com referencias no arquivo pdf, 
\usepackage[top=2.0cm, bottom=3.3cm, left=2.5cm,right=2.5cm]{geometry} % Margens
\usepackage{setspace}    					     % para colocar espaçamento entre linhas
\definecolor{seagreen}{RGB}{46,139,87}
\usepackage{scalefnt}
\usepackage{listings}

\lstset{
  extendedchars=\true,
  inputencoding=utf8,
  language=Matlab,
  %showstringspaces=false,
  formfeed=\newpage,
  tabsize=4,
  %commentstyle=\itshape,
  basicstyle=\ttfamily\scriptsize,
  %basicstyle={\small\fontfamily{fvm}\fontseries{m}\selectfont},
  commentstyle=\color{Apricot}\bfseries,
  %commentstyle=\color{red}\itshape,
  stringstyle=\color{red},
  identifierstyle=\color{PineGreen},
  showstringspaces=false,
  keywordstyle=\color{blue}\bfseries,
  moredelim=[il][\large\textbf]{\#\# },
  morekeywords={models,range},
  numbers=left,
  numbersep=2pt,
  numberstyle=\tiny,%\color{blue}\bfseries,
  literate=%
  {ã}{{\~a}}1
  {â}{{\^a}}1
  {õ}{{\~o}}1
  {á}{{\'a}}1
  {ú}{{\'u}}1
  {í}{{\'i}}1
  {é}{{\'e}}1
  {Ç}{{\c{C}}}1
  {Õ}{{\~O}}1
  {Ê}{{\^E}}1
  {ó}{{\'o}}1
  {à}{{\`a}}1
  {Â}{{\^A}}1
  {ô}{{\^o}}1
  {ê}{{\^e}}1
  {ç}{{\c{c}}}1
}
%%%%%%%%%%%%%%%%%%%%%%%%%%%%%%%%%%%%%%%%
% You have two versions of the macro
% \draftnote{My note}. The first version puts notes (e.g. My note in the example)
% into the margin of your document. The second formats the note as nothing. You
% 'comment out' the version of the macro you don't want (using a % at the
% beginning of the line).
\newcommand{\draftnote}[1]{\marginpar{\tiny\raggedright\textsf{\hspace{0pt}#1}}}
%\newcommand{\draftnote}[1]{}

% This one is just for the comments for in-line text.
\newcommand{\indraftnote}[1]{\textcolor[HTML]{114406}{\texttt{\footnotesize[#1]}}}
%\newcommand{\indraftnote}[1]{}
\newcommand{\todo}[1]{\indraftnote{todo: #1}}

%###########################################
%  outros comandos

\providecommand{\sin}{} \renewcommand{\sin}{\hspace{2pt}\mathrm{sen}}
\providecommand{\tan}{} \renewcommand{\tan}{\hspace{2pt}\mathrm{tg}}
\newcommand{\ie}{{\it i.e.}}
\newcommand{\etc}{{\it etc}}
\newcommand{\eg}{{\it e.g.}}
\newcommand{\keywordsname}{Keywords}
\newenvironment{keywords}{%
\noindent
        {\em \bfseries \keywordsname: }%
         \rm }
      {\endquotation \vskip 12bp}

\newcommand{\code}[2]{
 \vspace{1em}
 \subsubsection*{#1}
 \lstinputlisting{#2}
}
